\documentclass[letterpaper]{report}
\usepackage{amsmath}
\setlength{\parindent}{0pt}

\begin{document}
Solve the Equation:

\begin{equation}
\left(EA u_x \right)_x + \bar{f}Ax = 0
\end{equation}

For initial conditions:

\begin{enumerate}
\item $u(0) = g_1$, $u(L) = g_2$
\item $u(0) = g_1$, $u_x(L) = \frac{h}{EA}$
\end{enumerate}

For reference, the values are defined as:

\vspace{0.3in}
$E = 10^{11}~Pa$, $A = 10^{-4}~m^2$, $f = 10^{-4}~N$, $L = 0.1~m$

$g_1 = 0~m$, $g_2 = 0.001~m$, $h = 10^6~N$

\vspace{0.3in}
Solution to initial condition (1):

\begin{equation}
u(x) = -\frac{\bar{f}}{6E} x^3 + \left( \frac{g_2-g_1}{L} + \frac{\bar{f}L^2}{6E} \right) x + g_1
\end{equation}

Solution to initial condition (2):

\begin{equation}
u(x) = -\frac{\bar{f}}{6E} x^3 + \left( h+\frac{\bar{f}L^2}{2E} \right) x + g_1
\end{equation}

Lagrange Polynomial is used for the basis function in domain.

\begin{equation}
N^A(\xi) = \frac{\prod\limits_{\substack{B=1 \\ B\ne A}}^{N} (\xi-\xi_B)}{\prod\limits_{\substack{B=1 \\ B \ne A}}^{N} (\xi_A-\xi_B)}
\end{equation}

\end{document}